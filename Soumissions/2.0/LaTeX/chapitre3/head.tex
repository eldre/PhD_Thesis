\chapter{Description de l'exposition des écosystèmes benthiques côtiers aux multiples activités humaines à l'échelle locale}
\label{chap3}

\section{Résumé}
L'influence anthropique est un phénomène affectant tous les écosystèmes marins de la planète, dont la majorité est influencée par de multiples activités humaines. L'évaluation des impacts cumulés permet de comprendre comment les communautés et les habitats peuvent être affectés par des stresseurs d'origine anthropique, notamment grâce à l'étude de l'exposition et de la vulnérabilité des écosystèmes. De telles évaluations ont été développées à des échelles régionale ou mondiale afin de détecter des tendances à large échelle, et il peut être intéressant d'appliquer ces méthodes à fine résolution spatiale pour améliorer la gestion environnementale locale. Cette étude s'est intéressée à développer et appliquer un indice pour calculer l'exposition locale des écosystèmes à de multiples activités humaines. Les écosystèmes benthiques côtiers de la région de Sept-Îles (Québec, Canada) ont été sélectionnés, une zone portuaire industrielle dans le golfe du Saint-Laurent où de nombreuses activités humaines sont présentes. L'exposition a été calculée grâce à un modèle de diffusion particulaire et aux évènements de pêche dans la région, et les activités considérées ont été l'aquaculture, l'influence de la ville, l'influence des industries, le dragage des sédiments, la navigation commerciale, les égouts et les pêcheries. Une faible exposition a été détectée à l'échelle de la baie, avec des zones d'exposition cumulée devant la ville et les zones industrielles. Les modèles joints de distribution d'espèces ont détecté des relations significatives entre l'assemblage macrobenthique et des prédicteurs tels que les paramètres abiotiques et les indices d'exposition, permettant ainsi de rendre compte de la structure des communautés selon différents scénarios anthropiques. Cette étude présente des résultats intéressants sur les liens entre les activités humaines et les communautés benthiques à l'échelle locale, ouvrant la voie à des évaluations environnementales plus holistiques. \linebreak[4]

L'article associé à ce chapitre, "\textit{Describing exposure from multiple human activities on coastal benthic ecosystems at the local scale}", a été co-rédigé avec Rémi M Daigle, David Beauchesne, Julie Carrière, Christopher W. McKindsey et Philippe Archambault. Il est prévu de le soumettre dans la revue \textit{PLoS ONE} au courant de l'année 2021. J'ai établi les objectifs de ce chapitre avec Philippe Archambault. J'ai développé l'indice d'exposition avec Rémi M Daigle et David Beauchesne. Je me suis basé sur les données obtenues lors de la campagne d'échantillonnage en 2017 effectuée pour le premier chapitre, en collaboration avec Julie Carrière, pour étudier les liens entre communautés benthiques et indices d'exposition. J'ai dirigé la rédaction de l'article, où l'ensemble des co-auteurs a contribué à l'interprétation des résultats en fonction de leur expertise et à la révision générale. Certains résultats issus de ce chapitre ont été présentés lors de la Réunion Scientifique du \textit{Canadian Healthy Oceans Network} II à Ottawa en novembre 2018, la \textit{Global Change on Estuarine and Coastal Ecosystems Conference} (CHEERS) à Bordeaux en novembre 2019 et la Réunion Scientifique Annuelle de Québec-Océan à Québec en mars 2020. \linebreak[4]

\begin{singlespace}
Dreujou, E., Daigle, RM., Beauchesne, D., Carrière, J., McKindsey, CW., Archambault, P. (in prep). Describing exposure from multiple human activities on coastal benthic ecosystems at the local scale.
\end{singlespace}

\textit{Les sections suivantes sont celles de l'article en préparation.}

\clearpage

\begin{center}
\textbf{DESCRIBING EXPOSURE FROM MULTIPLE HUMAN ACTIVITIES ON COASTAL BENTHIC ECOSYSTEMS AT THE LOCAL SCALE}
\end{center}
