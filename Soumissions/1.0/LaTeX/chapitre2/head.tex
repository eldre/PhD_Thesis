\chapter{Déterminer le statut écologique de communautés benthiques côtières : le cas d'une zone industrielle portuaire canadienne}
\label{chap2}

\section{Résumé}
Compte tenu de l'influence généralisée des activités humaines sur les écosystèmes marins, l'évaluation de l'état écologique fournit des informations précieuses pour les initiatives de conservation et le développement durable. Ainsi, de nombreux indicateurs environnementaux ont été développés dans le monde et il est nécessaire d'évaluer leur performance en calculant l'état écologique dans une variété d'écosystèmes et à de multiples échelles spatio-temporelles. Cette étude a calculé et comparé seize indicateurs de l'état écologique, classés dans trois catégories méthodologiques: mesures d'abondance, paramètres de diversité et espèces caractéristiques. Cette sélection a été appliquée aux écosystèmes benthiques côtiers de Sept-Îles (Québec, Canada), une zone industrielle portuaire majeure dans le golfe du Saint-Laurent, et mise en relation avec les paramètres de l'habitat (matière organique, fractions granulométriques et concentrations de métaux lourds). Presque tous les indicateurs ont mis en évidence un état écologique généralement bon dans la zone d'étude, où les communautés présentaient un profil non-perturbé, avec une diversité élevée de taxons et de fonctions écosystèmiques, sans la dominance des taxons opportunistes. Plusieurs corrélations significatives avec les paramètres de l'habitat ont été détectées, en particulier avec les métaux lourds, et les analyses de rééchantillonnage ont détecté des résultats relativement solides. Cette étude fournit des renseignements précieux sur l'application d'indicateurs dans les écosystèmes côtiers canadiens, ainsi que sur leur utilisation à des fins d'évaluation environnementale. \linebreak[4]

L'article associé à ce chapitre, "\textit{Determining the ecological status of benthic coastal communities: a case study in a Canadian industrial harbour area}", a été rédigé en collaboration avec Nicolas Desroy, Lisa Tréau de Coeli, Julie Carrière, Christopher W. McKindsey et Philippe Archambault. Il a été soumis dans la revue \textit{Frontiers in Marine Science}, dans la section spéciale \textit{Biodiversity and Distribution of Benthic Invertebrates - From Taxonomy to Ecological Patterns and Global Processes}, le 3 décembre 2020. J'ai établi les objectifs de ce chapitre avec Nicolas Desroy, Christopher W. McKindsey et Philippe Archambault. Je me suis basé sur les données obtenues lors de la campagne d'échantillonnage en 2017 effectuée pour le premier chapitre, en collaboration avec Julie Carrière, auxquelles j'ai ajouté des données sur les traits biologiques collectées depuis différentes bases de données en ligne, validées par Lisa Tréau de Coeli. J'ai calculé les indicateurs environnementaux au cours d'un stage à la station biologique de Dinard avec Nicolas Desroy, et j'ai ensuite effectué les analyses statistiques pour évaluer et comparer les résultats des différents indicateurs. J'ai dirigé la rédaction de l'article, où l'ensemble des co-auteurs a contribué à l'interprétation des résultats en fonction de leur expertise et à la révision générale. Les données liées à cet article sont accessibles dans le dépôt en ligne hébergé par le site Scholars Portal Dataverse avec l'identifiant unique [10.5683/SP2/WDDDMI](https://doi.org/10.5683/SP2/WDDDMI). Une partie des résultats de ces analyses a été présentée lors de la Réunion Scientifique du \textit{Canadian Healthy Oceans Network} II à Ottawa en novembre 2018. \linebreak[4]

\begin{singlespace}
Dreujou E, Desroy N, Carrière J, Tréau de Coeli L, McKindsey CW, Archambault P (submitted). Determining the ecological status of benthic coastal communities: a case study in a Canadian industrial harbour area. \textit{Frontiers in Marine Science}.
\end{singlespace}

\textit{Les sections suivantes sont celles de l’article soumis, en révision.}
