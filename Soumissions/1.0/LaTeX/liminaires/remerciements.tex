Si je suis ici, au Canada, à écrire une thèse de doctorat, c'est grâce
au soutien de mes proches qui m'ont toujours poussé à réaliser mes
projets et à me dépasser. Je n'aurais jamais pu accomplir ce travail,
fruit de cinq années de recherche, sans votre présence, vos conseils,
vos réflexions, et je souhaite vous en remercier.

Merci à mon équipe de direction. Phil, je t'ai contacté pour la première
fois quand j'étais à la fin de ma maîtrise, à un moment où je ne savais
pas si j'étais suffisamment bon pour la recherche, et tu m'as fait
confiance. Ensemble, nous avons construit un beau projet, de Rimouski à
Québec, et je suis fier d'avoir pu profiter de ton expertise pour
éclairer ma voie. Tu m'as fait aller plus haut et plus loin que ce que
je n'aurais jamais pu imaginer, et j'ai hâte de poursuivre mon envol à
tes côtés. Chris, merci d'avoir accepté d'être mon codirecteur. Tes avis
m'ont été particulièrement utiles, et ont permis de pousser mes
reflexions toujours plus loin, tout comme la qualité de mes articles.

Merci aux réseaux de recherche et structures qui ont rendu ce projet
possible par leur financement, leur aide et leur expertise :
\emph{Canadian Healthy Oceans Network} II, Québec-Océan, Institut
Nordique de Recherche en Environnement et en Santé au Travail, Takuvik,
Réseau Québec Maritime et Institut France-Québec Maritime. Les réunions
scientifiques, ateliers de travail et écoles d'été auxquels j'ai pu
participer m'ont apporté de nombreuses compétences inestimables, pour
mener à bien mes projets mais aussi pour développer mon réseau. Je suis
très fier d'avoir été élu aux comités étudiants, cela a été une
expérience enrichissante et un plaisir de représenter les étudiants
ainsi que de mettre en place de nouvelles activités. Je remercie aussi
chaleureusement Natalie Ban, Nicolas Desroy et Aurélie Foveau de m'avoir
accueilli lors de stages de recherche, m'apportant la possibilité
d'aggrandir la portée de ma recherche. Enfin, je remercie l',

Merci à mes amis et collègues, sans qui cette aventure aurait été bien
moins belle. David, je ne te remercierai jamais assez pour ton mentorat,
d'avoir pris le temps de me faire découvrir de nouveaux outils qui me
sont aujourd'hui indispensables, d'avoir répondu à mes (INNOMBRABLES)
questions, et de m'avoir fait découvrir le bourbon, surtout après une
bonne journée de terrain. Charlotte, tu es la joie incarnée et ton
soutien a été exceptionnel. Je ne compte plus les \emph{amaretto sour}
que j'ai bu grace à toi (encore de l'alcool, décidément), et j'ai hâte
de jouer à un nouveau jeu avec toi ! Marie, tu as apporté du bonheur
dans mes journées de travail, je ne me lasserai jamais de l'envoi de sel
(et de paquets de mayonnaise) à travers le bureau, et je suis heureux
d'avoir découvert Québec à tes côtés (ainsi que Mushu). Laurie et
Valérie, merci de m'avoir aidé à traverser ces épreuves et pour nos
éclats de rire d'au moins trois ``Elliot'' ! Jesi, tu es la première
personne qui m'a accueilli quand je suis arrivé à Rimouski par bus, avec
mes valises à une heure beaucoup trop tardive. Merci pour tout (et je te
dois toujours un alfajor en retour !). Enfin, je n'aurais jamais assez
de temps pour remercier PO, Fanny, Gustavo, Jean-Luc, Inge, Sarah,
Déborah, Rémi, Gwenaëlle, Marie, Jory, Clémence et Piero pour les
moments inoubliables que nous avons partagé, ainsi que Cindy, Lisa,
Laure, Julie, Claudy, Serge, Dany, Martine, Marièle et Dominique sans
qui je me serais retrouvé bloqué à de nombreuses reprises.

Merci à mes collocs à Rimouski, dans la grande et belle Maison Jaune au
bord du Saint-Laurent. Camille, Francis, Antoine, Léo, ainsi que Éric et
Hélène, vous m'avez fait vivre des soirées incroyables (TUMMIES !!!) et
je garderai un doux souvenir de ces belles journées de cuisine, de Mario
Kart, Rocket League et surtout du \emph{chill} au bord du fleuve.
J'aimerais aussi remercier les créateurs qui m'ont fait rire, réfléchir,
explorer, découvrir et garder une certaine santé mentale grâce à leurs
productions : Frédéric Molas, Sébastien Rassiat, Mahyar Shakeri, Bob
Lennon, Patrick Baud, Gull et son technicien, Léo Grasset, Bruce
Benamran, Vivianne Lalande et les frères Breut.

Enfin, je voudrais remercier ma famille. Annick et Jean-Marie, mon papa
et ma maman, vous m'avez toujours épaulé à chaque étape de ma vie, de
mes études à Paris jusqu'à mon départ pour le Canada. Vous m'avez permis
d'être qui je suis aujourd'hui, et je n'aurai jamais assez de mots pour
vous dire à quel point je vous aime. Arthur, tu es mon p'tit frérôt, mon
sale type préféré, mon partenaire de jeu, de dessin, de zones, de
construction, de tornades et de dances ridicules. Je ne serai pas aussi
heureux et épanouit aujourd'hui sans avoir grandi à tes côtés. Sara. Mon
amour, ma truffe, ma Schouffe adorée. ``Marullette''. Je n'aurais jamais
réussi à tenir le coup sans toi. Paris, Rimouski, Québec, Montréal, tu
as toujours été ma complice. Tu m'as supporté dans mes pires moments, tu
m'as rendu fier et fort, et tu m'as enrichi de ton intelligence, ton
humour (\emph{ech!}), tes carabistouilles et de ton amour. Je suis le
plus heureux des hommes, depuis le jour où j'ai goûté ce
\textbf{fantastique} risotto, et mes journées ne pourraient être plus
belles qu'à tes côtés. Je t'aime ma Sara.

Je souhaite aussi adresser mes derniers mots à mon oncle Alex, qui nous
a quitté bien trop tôt. Tu me manques.
