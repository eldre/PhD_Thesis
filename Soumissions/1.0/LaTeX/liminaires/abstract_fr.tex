L'ensemble des environnements côtiers et océaniques de la planète est
influencé par les activités humaines, dont les impacts peuvent modifier
la structure et l'intégrité des écosystèmes de façon durable. Afin de
protéger adéquatement le milieu naturel et de soutenir un développement
anthropique durable, notamment dans des régions concentrant de multiples
activités humaines, il est nécessaire de comprendre comment les
écosystèmes marins sont influencés. Les communautés benthiques sont un
compartiment particulièrement intéressant pour étudier ces
problématiques, car de nombreuses espèces possèdent des capacités
d'évitement limitées de par leur mode de vie majoritairement sessile
ainsi qu'une espérance de vie élevée. Alors que travaux évaluant
l'impact anthropique ont été effectués sur une large gamme d'écosystèmes
à travers le monde, peu ont considéré spécifiquement des écosystèmes
subarctiques, dont l'utilisation par l'homme est prévue d'augmenter en
lien avec le changement climatique. C'est dans ce contexte que s'inscrit
ma thèse, où l'objectif principal est de comprendre comment les
écosystèmes benthiques d'une zone industrialo-portuaire subarctique sont
influencés par les activités humaines. La zone d'étude considérée se
trouve dans la région de Sept-Îles (Québec, Canada), plateforme
économique importante pour le Québec, située dans le Golfe du
Saint-Laurent. Pour répondre à ces problématiques, cette thèse est
divisée en trois chapitres.

Le premier chapitre a pour but de caractériser la structure des
écosystèmes côtiers considérés. Lors de campagnes de terrain et
d'analyses en laboratoire, un total de 289 taxons a été échantillonné,
dont la majorité, présentes dans le Golfe du Saint-Laurent, sont des
nouvelles mentions dans cette région, et divers paramètres abiotiques du
sédiment ont été évalués, telles que la concentration en matière
organique, en métaux lourds et la distribution de fractions
granulométriques. L'analyse de la similarité des assemblages
d'invertébrés de taille supérieure à 0.5 mm a détecté des signes de
perturbation dans certaines zones, avec un nombre accru d'espèces
tolérant la pollution et d'espèces opportunistes. Des modèles de
régression ont permis de mettre en évidence les variables de l'habitat
qui impactent le plus la structure des communautés.

Le deuxième chapitre s'intéresse au statut écologique des écosystèmes en
se basant sur la composition des communautés benthiques. Seize
indicateurs du statut écologique ont été sélectionnés au moyen d'une
revue de littérature, divisés en trois catégories selon leur
méthodologie : mesures d'abondance, diversité des communautés et espèces
indicatrices. Ces indicateurs ont été appliqués en utilisant les listes
d'espèces obtenues lors du chapitre précédent, et la majorité ont
détecté des communautés diversifiées sans signe évident de perturbation.
De plus, plusieurs corrélations significatives ont été détectées entre
les indicateurs et les paramètres de l'habitat, notamment avec les
concentrations en métaux lourds. Chaque catégorie d'indicateur apporte
des informations importantes sur l'état de l'écosystème tout en
présentant des limitations, en particulier à propos des références
utilisées pour définir le statut écologique.

Le dernier chapitre a considéré les activités humaines influençant
l'écosystème, afin de calculer une empreinte anthropique locale sur les
communautés selon des gradients d'exposition. Un modèle de diffusion
pour chaque activité considérée (aquaculture, dragage, influence
industrielle, influence municipale, pêcheries, rejets d'égouts,
transport maritime) a été développé grâce à la distance depuis leur(s)
source(s) et des facteurs physiques. Plusieurs liens ont été découverts
entre les indices d'expositions obtenus et la distribution des
invertébrés benthiques, au moyen de modèles prédictifs
\textit{Hierarchical Modelling of Species Communities}. L'indice
d'exposition cumulée a mis en évidence des zones de superposition
d'activité humaine. Le profil des communautés présentes dans ces régions
n'est pas particulièrement perturbé, ce qui corrobore les résultats des
chapitres précédents sur le statut des écosystèmes considérés.

Cette thèse de doctorat contribue à l'amélioration des connaissances sur
les écosystèmes côtiers subarctiques, notamment en présentant la
première étude de biodiversité benthique dans la région de Sept-Îles.
Des méthodes d'évaluation du statut écologique et de l'exposition
anthropique ont été développées à l'échelle locale (\textless{} 100 km),
qui constituent des outils particulièrement intéressants pour les
gestionnaires afin de définir des objectifs de gestion et de soutenir
des initiatives de conservation.

\begin{quote}
Mots clefs : écologie marine, écosystèmes côtiers subarctiques,
invertébrés benthiques, biodiversité, prédiction des communautés,
activités humaines, exposition anthropique, évaluation du statut
écologique.
\end{quote}
