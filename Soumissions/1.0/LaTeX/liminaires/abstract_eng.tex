Coastal and ocean environments are influenced by human activities
worldwide, the impacts of which can significantly modify the structure
and integrity of ecosystems. In order to adequately protect the natural
environment and support sustainable anthropogenic development,
particularly in regions where multiple human activities cooccur, it is
necessary to understand how marine ecosystems are influenced. Benthic
communities are a particularly interesting compartment for studying
these issues, because many species have a limited mobility due to their
predominantly sessile lifestyle as well as a long life span. While
studies assessing anthropogenic impact have been carried out on a wide
range of ecosystems around the world, few have specifically considered
sub-Arctic ecosystems, where human activity is expected to increase in
connection with climate change. In this context, my thesis' main
objective is to understand how the benthic ecosystems of a sub-Arctic
industrial harbour area are influenced by human activities. The study
area considered is located in the Sept-Îles region (Quebec, Canada), an
important economic hub for Quebec, located in the Gulf of St.~Lawrence.
To address these issues, this thesis is divided into three chapters.

The first chapter aims to characterize the structure of the considered
coastal ecosystems. During field campaigns and laboratory analyses, a
total of 289 taxa were sampled, the majority of which, present in the
Gulf of St.Lawrence, are new records in this region, and various abiotic
parameters of the sediment were assessed, such as the concentration of
organic matter, heavy metals and the distribution of particle size
fractions. Similarity analysis of invertebrate assemblages larger than
0.5 mm detected signs of disturbance in some areas, with an increased
number of pollution-tolerant and opportunistic species. Regression
models highlighted which habitat variables have the most impact on the
structure of communities.

The second chapter looks at the ecological status of ecosystems based on
the composition of benthic communities. Sixteen indicators of ecological
status were selected through a literature review, divided into three
categories according to their methodology: measures of abundance,
community diversity and indicator species. These indicators were applied
using the species lists obtained in the previous chapter, and the
majority detected diverse communities with no obvious sign of
disturbance. In addition, several significant correlations were detected
between indicators and habitat parameters, especially with heavy metal
concentrations. Each category of indicator provides important
information on the state of the ecosystem while presenting limitations,
in particular about reference conditions used to define ecological
status.

The last chapter considered human activities influencing the ecosystem,
in order to calculate a local anthropogenic footprint on communities
according to exposure gradients. A particle diffusion model for each
activity considered (aquaculture, dredging, industrial influence, city
influence, fisheries, sewage discharges, shipping) was developed using
the distance from their sources and physical factors. Several links were
discovered between the exposure indices obtained and the distribution of
benthic invertebrates, using predictive models
\textit{Hierarchical Modeling of Species Communities}. The cumulative
exposure index revealed areas of superposition of human activity. The
profile of the communities present in these zones is not particularly
disturbed, which corroborates the results of the previous chapters on
the status of the ecosystems considered.

This PhD thesis contributes to improving ecological knowledge in
sub-Arctic coastal ecosystems, in particular by presenting the first
benthic biodiversity census in the Sept-Îles region. Methods for
assessing ecological status and anthropogenic exposure have been
developed at the local scale (\textless100 km), which constitute
particularly interesting tools for managers in order to define
management targets and support conservation initiatives.

\begin{quote}
Keywords: marine ecology, sub-Arctic coastal ecosystems, benthic
invertebrates, biodiversity, community prediction, human activities,
anthropogenic exposure, ecological status assessment.
\end{quote}
