\appendice{Approches environnementales holistiques et objectifs d'Aichi pour la biodiversité : accomplissements et perspectives pour les écosystèmes marins}
\label{ann1}
\addtocounter{chapter}{1}

\section{Résumé}
Afin de préserver la biodiversité des changements mondiaux, la Conférence des Parties a élaboré un Plan Stratégique pour la Biodiversité pour la période 2011-2020 qui comprenait une liste de vingt objectifs spécifiques connus sous le nom d'objectifs d'Aichi pour la biodiversité. Cette période arrivant à sa fin, et malgré des progrès majeurs dans la conservation de la biodiversité, les preuves suggèrent que la majorité des objectifs ne seront probablement pas atteints. Cet article fait partie d'une série d'articles de perspective en lien avec la 4\textsuperscript{ème} \textit{World Conference on Marine Biodiversity} (mai 2018, Montréal, Canada) pour identifier les prochaines étapes vers une conservation réussie de la biodiversité en milieu marin. Nous avons considéré les études avec une approche environnementale holistique (AEH) et leur contribution à l'atteinte des objectifs. Notre analyse est axée sur différentes approches environnementales pouvant être considérées comme holistiques, et nous discutons de la manière dont les AEH peut contribuer aux objectifs d'Aichi pour la biodiversité à l'avenir. Nous avons constaté que seuls quelques AEH considéraient un objectif de biodiversité spécifique dans leurs recherches, et que l'objectif 11, qui porte sur les aires marines protégées, était le plus souvent cité. Nous proposons cinq priorités de recherche pour améliorer les AEH pour la conservation de la biodiversité marine au-delà de 2020 : (i) étendre l'utilisation d'approches holistiques dans les évaluations environnementales, (ii) normaliser le vocabulaire utilisé par les AEH, (iii) améliorer la collecte, le partage et la gestion des données, (iv) envisager la variabilité spatio-temporelle des écosystèmes et (v) intégrer les services écosystémiques dans les AEH. La prise en compte de ces priorités favorisera la valeur des AEH et profitera au Plan Stratégique pour la Biodiversité.

L'article associé à ce chapitre, "\textit{Holistic Environmental Approaches and Aichi Biodiversity Targets: accomplishments and perspectives for marine ecosystems}", a été rédigé en collaboration avec Charlotte Carrier-Belleau, Jesica Goldsmit, Dario Fiorentino, Radhouane Ben-Hamadou, Jose H Muelbert, Jasmin A Godbold, Rémi M Daigle et David Beauchesne. Il a été publié dans le journal \textit{PeerJ}, le 25 février 2020. Cet article d'opinion a été réalisé dans le cadre du programme de mentorat organisé lors de la 4\textsuperscript{ème} \textit{World Conference on Marine Biodiversity}. J'ai dirigé l'équipe de rédaction, composée d'étudiants et de chercheurs (mentors scientifiques), pendant et après la conférence pour déterminer la direction des discussions, identifier les priorités de recherche et d'en discuter les implications, en lien avec les objectifs définis avec Jesica Goldsmit, Rémi M Daigle et David Beauchesne. J'ai effectué la revue de littérature, et j'ai dirigé la rédaction de l'article, où l'ensemble des co-auteurs a contribué à l'écriture des priorités de recherche et à la révision générale. \linebreak[4]

\begin{singlespace}
Dreujou E, Carrier-Belleau C, Goldsmit J, Fiorentino D, Ben-Hamadou R, Muelbert JH, Godbold JA, Daigle RM, Beauchesne D (2020). Holistic Environmental Approaches and Aichi Biodiversity Targets: accomplishments and perspectives for marine ecosystems. \textit{PeerJ} 8:e817. DOI:10.7717/peerj.8171.
\end{singlespace}

\textit{Les sections suivantes correspondent à celles de l'article publié.}
