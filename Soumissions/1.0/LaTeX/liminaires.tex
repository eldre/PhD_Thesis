%----------------------------------------------------------------------%
% Liminaires de la thèse.                                              %
% UQAR septembre 2013                                                  %
% ---------------------------------------------------------------------%

% ----------------------------------------------------------------------%
% 1- Page titre.                                                        %
% ----------------------------------------------------------------------%

\Pagetitre
\cleardoublepage
% ----------------------------------------------------------------------%
% inclusions qui pourraient mériter d'être incluses dans le .cls
% (commentez si non-nécessaire)
% 1.1 - Composition du Jury.                                           %
\thispagestyle{empty}

\null
\vfill
\noindent \textbf{Composition du jury:}\\
\vspace{1cm}

\begin{singlespace}
  \noindent \textbf{QQ1, président du jury, Université du Québec à Rimouski}\\

  \noindent \textbf{Phillipe Archambault, directeur de recherche, Université Laval}\\

  \noindent \textbf{Dominique Gravel, codirecteur de recherche, Université de Sherbrooke}\\

  \noindent \textbf{Jean-Claude Brêthes, examinateur interne, Université du Québec à Rimouski}\\

  \noindent \textbf{QQ1, examinateur externe, QQpart}\\
\end{singlespace}

\vspace{2cm}
\noindent Dépôt initial le 10 décembre 2019
\hspace{3cm}
Dépôt final le 10 décembre 2019


\cleardoublepage

% % 1.2 - Avertissement biblio.
\thispagestyle{empty}

\vspace{2cm}
\begin{center}
UNIVERSITÉ DU QUÉBEC À RIMOUSKI\\
Service de la bibliothèque
\end{center}

\vspace{3cm}
\begin{center}
Avertissement
\end{center}


\vspace{1cm}

\noindent La diffusion de ce mémoire ou de cette thèse se fait dans le respect des droits de son auteur, qui a signé le formulaire {\itshape \og Autorisation de reproduire et de diffuser un rapport, un mémoire ou une thèse \fg}. 
En signant ce formulaire, l’auteur concède à l’Université du Québec à Rimouski une licence non exclusive d’utilisation et de publication de la totalité ou d’une partie importante de son travail de recherche pour des fins pédagogiques et non commerciales. 
Plus précisément, l’auteur autorise l’Université du Québec à Rimouski à reproduire, diffuser, prêter, distribuer ou vendre des copies de son travail de recherche à des fins non commerciales sur quelque support que ce soit, y compris l’Internet. 
Cette licence et cette autorisation n’entraînent pas une renonciation de la part de l’auteur à ses droits moraux ni à ses droits de propriété intellectuelle. 
Sauf entente contraire, l’auteur conserve la liberté de diffuser et de commercialiser ou non ce travail dont il possède un exemplaire.



\cleardoublepage
% % 1.3 - Dedicace.
\thispagestyle{empty}

\begin{minipage}[l]{0.45\textwidth}

\end{minipage}%
\hfill
\begin{minipage}[r]{0.5\textwidth}
\begin{quotation}
\begin{doublespace}

\guillemotleft \textit{Oui, la mer est tout. Je l'aime ! Elle couvre les sept dixièmes du globe terrestre. Son souffle est pur et sain. C'est l'immense désert où l'homme n'est jamais seul, car il sent frémir la vie à ses côtés.} \\
\textit{Ah ! monsieur, vivez, vivez au sein des mers. Là seulement est l'indépendance ! Là, je ne reconnais pas de maîtres ; là, je suis libre !} \guillemotright \\
Vingt mille lieues sous les mers, \\
\textbf{Jules Verne}

\vspace{15mm}

\guillemotleft \textit{ C'est dans l'action que l'on reconnait les champions.} \guillemotright \\
Slogan d'une enseigne rencontrée tous les jours à Sept-Îles, \\
\textbf{Inconnu}

\end{doublespace}
\end{quotation}
\end{minipage}%

\cleardoublepage

% ----------------------------------------------------------------------%


% ----------------------------------------------------------------------%
% 2- Remerciements.                                                    %
% ----------------------------------------------------------------------%

\remerciements
% \selectlanguage{french}
\selectlanguage{english}
Si je suis ici, au Canada, à écrire une thèse de doctorat, c'est grâce
au soutien de mes proches qui m'ont toujours poussé à réaliser mes
projets et à me dépasser. Je n'aurais jamais pu accomplir ce travail,
fruit de cinq années de recherche, sans votre présence, vos conseils,
vos réflexions, et je souhaite vous en remercier.

Merci à mon équipe de direction. Phil, je t'ai contacté pour la première
fois quand j'étais à la fin de ma maîtrise, à un moment où je ne savais
pas si j'étais suffisamment bon pour la recherche, et tu m'as fait
confiance. Ensemble, nous avons construit un beau projet, de Rimouski à
Québec, et je suis fier d'avoir pu profiter de ton expertise pour
éclairer ma voie. Tu m'as fait aller plus haut et plus loin que ce que
je n'aurais jamais pu imaginer, et j'ai hâte de poursuivre mon envol à
tes côtés. Chris, merci d'avoir accepté d'être mon codirecteur. Tes avis
m'ont été particulièrement utiles, et ont permis de pousser mes
reflexions toujours plus loin, tout comme la qualité de mes articles.

Merci aux réseaux de recherche et structures qui ont rendu ce projet
possible par leur financement, leur aide et leur expertise :
\emph{Canadian Healthy Oceans Network} II, Québec-Océan, Institut
Nordique de Recherche en Environnement et en Santé au Travail, Takuvik,
Réseau Québec Maritime et Institut France-Québec Maritime. Les réunions
scientifiques, ateliers de travail et écoles d'été auxquels j'ai pu
participer m'ont apporté de nombreuses compétences inestimables, pour
mener à bien mes projets mais aussi pour développer mon réseau. Je suis
très fier d'avoir été élu aux comités étudiants, cela a été une
expérience enrichissante et un plaisir de représenter les étudiants
ainsi que de mettre en place de nouvelles activités. Je remercie aussi
chaleureusement Natalie Ban, Nicolas Desroy et Aurélie Foveau de m'avoir
accueilli lors de stages de recherche, m'apportant la possibilité
d'aggrandir la portée de ma recherche. Enfin, je remercie l',

Merci à mes amis et collègues, sans qui cette aventure aurait été bien
moins belle. David, je ne te remercierai jamais assez pour ton mentorat,
d'avoir pris le temps de me faire découvrir de nouveaux outils qui me
sont aujourd'hui indispensables, d'avoir répondu à mes (INNOMBRABLES)
questions, et de m'avoir fait découvrir le bourbon, surtout après une
bonne journée de terrain. Charlotte, tu es la joie incarnée et ton
soutien a été exceptionnel. Je ne compte plus les \emph{amaretto sour}
que j'ai bu grace à toi (encore de l'alcool, décidément), et j'ai hâte
de jouer à un nouveau jeu avec toi ! Marie, tu as apporté du bonheur
dans mes journées de travail, je ne me lasserai jamais de l'envoi de sel
(et de paquets de mayonnaise) à travers le bureau, et je suis heureux
d'avoir découvert Québec à tes côtés (ainsi que Mushu). Laurie et
Valérie, merci de m'avoir aidé à traverser ces épreuves et pour nos
éclats de rire d'au moins trois ``Elliot'' ! Jesi, tu es la première
personne qui m'a accueilli quand je suis arrivé à Rimouski par bus, avec
mes valises à une heure beaucoup trop tardive. Merci pour tout (et je te
dois toujours un alfajor en retour !). Enfin, je n'aurais jamais assez
de temps pour remercier PO, Fanny, Gustavo, Jean-Luc, Inge, Sarah,
Déborah, Rémi, Gwenaëlle, Marie, Jory, Clémence et Piero pour les
moments inoubliables que nous avons partagé, ainsi que Cindy, Lisa,
Laure, Julie, Claudy, Serge, Dany, Martine, Marièle et Dominique sans
qui je me serais retrouvé bloqué à de nombreuses reprises.

Merci à mes collocs à Rimouski, dans la grande et belle Maison Jaune au
bord du Saint-Laurent. Camille, Francis, Antoine, Léo, ainsi que Éric et
Hélène, vous m'avez fait vivre des soirées incroyables (TUMMIES !!!) et
je garderai un doux souvenir de ces belles journées de cuisine, de Mario
Kart, Rocket League et surtout du \emph{chill} au bord du fleuve.
J'aimerais aussi remercier les créateurs qui m'ont fait rire, réfléchir,
explorer, découvrir et garder une certaine santé mentale grâce à leurs
productions : Frédéric Molas, Sébastien Rassiat, Mahyar Shakeri, Bob
Lennon, Patrick Baud, Gull et son technicien, Léo Grasset, Bruce
Benamran, Vivianne Lalande et les frères Breut.

Enfin, je voudrais remercier ma famille. Annick et Jean-Marie, mon papa
et ma maman, vous m'avez toujours épaulé à chaque étape de ma vie, de
mes études à Paris jusqu'à mon départ pour le Canada. Vous m'avez permis
d'être qui je suis aujourd'hui, et je n'aurai jamais assez de mots pour
vous dire à quel point je vous aime. Arthur, tu es mon p'tit frérôt, mon
sale type préféré, mon partenaire de jeu, de dessin, de zones, de
construction, de tornades et de dances ridicules. Je ne serai pas aussi
heureux et épanouit aujourd'hui sans avoir grandi à tes côtés. Sara. Mon
amour, ma truffe, ma Schouffe adorée. ``Marullette''. Je n'aurais jamais
réussi à tenir le coup sans toi. Paris, Rimouski, Québec, Montréal, tu
as toujours été ma complice. Tu m'as supporté dans mes pires moments, tu
m'as rendu fier et fort, et tu m'as enrichi de ton intelligence, ton
humour (\emph{ech!}), tes carabistouilles et de ton amour. Je suis le
plus heureux des hommes, depuis le jour où j'ai goûté ce
\textbf{fantastique} risotto, et mes journées ne pourraient être plus
belles qu'à tes côtés. Je t'aime ma Sara.

Je souhaite aussi adresser mes derniers mots à mon oncle Alex, qui nous
a quitté bien trop tôt. Tu me manques.


% [Cette page est facultative; l’éliminer si elle n’est pas utilisée. Les remerciements peuvent aussi être intégrés à l'avant-propos. C’est dans cette section que l’on remercie les personnes qui ont contribué au projet, les organismes ou les entreprises subventionnaires qui ont soutenu financièrement le projet.]



% ----------------------------------------------------------------------%
% 3- Avant-propos.                                                     %
% ----------------------------------------------------------------------%

\avantpropos
% \selectlanguage{french}
\selectlanguage{english}
La j'ai fait des merci formels!



% [Cette page est facultative; l’éliminer si elle n’est pas utilisée. L’avant-propos ne doit pas être confondu avec l'introduction. Il n’est pas d’ordre scientifique alors que l’introduction l’est. Il s’agit d'un discours préliminaire qui permet notamment à l'auteur d'exposer les raisons qui l'ont amené à étudier le sujet choisi, le but qu'il veut atteindre, ainsi que les possibilités et les limites de son travail. On peut inclure les remerciements à la fin de ce texte au lieu de les présenter sur une page distincte.]



% ----------------------------------------------------------------------%
% 4- Resume/Abstract                                                           %
% ----------------------------------------------------------------------%

\resume
\begin{singlespace}
L'ensemble des environnements côtiers et océaniques de la planète est
influencé par les activités humaines, dont les impacts peuvent modifier
la structure et l'intégrité des écosystèmes de façon durable. Afin de
protéger adéquatement le milieu naturel et de soutenir un développement
anthropique durable, notamment dans des régions concentrant de multiples
activités humaines, il est nécessaire de comprendre comment les
écosystèmes marins sont influencés. Les communautés benthiques sont un
compartiment particulièrement intéressant pour étudier ces
problématiques, car de nombreuses espèces possèdent des capacités
d'évitement limitées de par leur mode de vie majoritairement sessile
ainsi qu'une espérance de vie élevée. Alors que travaux évaluant
l'impact anthropique ont été effectués sur une large gamme d'écosystèmes
à travers le monde, peu ont considéré spécifiquement des écosystèmes
subarctiques, dont l'utilisation par l'homme est prévue d'augmenter en
lien avec le changement climatique. C'est dans ce contexte que s'inscrit
ma thèse, où l'objectif principal est de comprendre comment les
écosystèmes benthiques d'une zone industrialo-portuaire subarctique sont
influencés par les activités humaines. La zone d'étude considérée se
trouve dans la région de Sept-Îles (Québec, Canada), plateforme
économique importante pour le Québec, située dans le Golfe du
Saint-Laurent. Pour répondre à ces problématiques, cette thèse est
divisée en trois chapitres.

Le premier chapitre a pour but de caractériser la structure des
écosystèmes côtiers considérés. Lors de campagnes de terrain et
d'analyses en laboratoire, un total de 289 taxons a été échantillonné,
dont la majorité, présentes dans le Golfe du Saint-Laurent, sont des
nouvelles mentions dans cette région, et divers paramètres abiotiques du
sédiment ont été évalués, telles que la concentration en matière
organique, en métaux lourds et la distribution de fractions
granulométriques. L'analyse de la similarité des assemblages
d'invertébrés de taille supérieure à 0.5 mm a détecté des signes de
perturbation dans certaines zones, avec un nombre accru d'espèces
tolérant la pollution et d'espèces opportunistes. Des modèles de
régression ont permis de mettre en évidence les variables de l'habitat
qui impactent le plus la structure des communautés.

Le deuxième chapitre s'intéresse au statut écologique des écosystèmes en
se basant sur la composition des communautés benthiques. Seize
indicateurs du statut écologique ont été sélectionnés au moyen d'une
revue de littérature, divisés en trois catégories selon leur
méthodologie : mesures d'abondance, diversité des communautés et espèces
indicatrices. Ces indicateurs ont été appliqués en utilisant les listes
d'espèces obtenues lors du chapitre précédent, et la majorité ont
détecté des communautés diversifiées sans signe évident de perturbation.
De plus, plusieurs corrélations significatives ont été détectées entre
les indicateurs et les paramètres de l'habitat, notamment avec les
concentrations en métaux lourds. Chaque catégorie d'indicateur apporte
des informations importantes sur l'état de l'écosystème tout en
présentant des limitations, en particulier à propos des références
utilisées pour définir le statut écologique.

Le dernier chapitre a considéré les activités humaines influençant
l'écosystème, afin de calculer une empreinte anthropique locale sur les
communautés selon des gradients d'exposition. Un modèle de diffusion
pour chaque activité considérée (aquaculture, dragage, influence
industrielle, influence municipale, pêcheries, rejets d'égouts,
transport maritime) a été développé grâce à la distance depuis leur(s)
source(s) et des facteurs physiques. Plusieurs liens ont été découverts
entre les indices d'expositions obtenus et la distribution des
invertébrés benthiques, au moyen de modèles prédictifs
\textit{Hierarchical Modelling of Species Communities}. L'indice
d'exposition cumulée a mis en évidence des zones de superposition
d'activité humaine. Le profil des communautés présentes dans ces régions
n'est pas particulièrement perturbé, ce qui corrobore les résultats des
chapitres précédents sur le statut des écosystèmes considérés.

Cette thèse de doctorat contribue à l'amélioration des connaissances sur
les écosystèmes côtiers subarctiques, notamment en présentant la
première étude de biodiversité benthique dans la région de Sept-Îles.
Des méthodes d'évaluation du statut écologique et de l'exposition
anthropique ont été développées à l'échelle locale (\textless{} 100 km),
qui constituent des outils particulièrement intéressants pour les
gestionnaires afin de définir des objectifs de gestion et de soutenir
des initiatives de conservation.

\begin{quote}
Mots clefs : écologie marine, écosystèmes côtiers subarctiques,
invertébrés benthiques, biodiversité, prédiction des communautés,
activités humaines, exposition anthropique, évaluation du statut
écologique.
\end{quote}

  % [Le résumé en français doit présenter en 350 mots maximum pour un mémoire et en 700 mots pour une thèse : (1) le but de la recherche, (2) les sujets étudiés, (3) les hypothèses de travail et la méthode utilisée, (4) les principaux résultats et (5) les conclusions de l'étude ou de la recherche.]

\end{singlespace}
\cleardoublepage


\abstract
\begin{singlespace}
Coastal and ocean environments are influenced by human activities
worldwide, the impacts of which can significantly modify the structure
and integrity of ecosystems. In order to adequately protect the natural
environment and support sustainable anthropogenic development,
particularly in regions where multiple human activities cooccur, it is
necessary to understand how marine ecosystems are influenced. Benthic
communities are a particularly interesting compartment for studying
these issues, because many species have a limited mobility due to their
predominantly sessile lifestyle as well as a long life span. While
studies assessing anthropogenic impact have been carried out on a wide
range of ecosystems around the world, few have specifically considered
sub-Arctic ecosystems, where human activity is expected to increase in
connection with climate change. In this context, my thesis' main
objective is to understand how the benthic ecosystems of a sub-Arctic
industrial harbour area are influenced by human activities. The study
area considered is located in the Sept-Îles region (Quebec, Canada), an
important economic hub for Quebec, located in the Gulf of St.~Lawrence.
To address these issues, this thesis is divided into three chapters.

The first chapter aims to characterize the structure of the considered
coastal ecosystems. During field campaigns and laboratory analyses, a
total of 289 taxa were sampled, the majority of which, present in the
Gulf of St.Lawrence, are new records in this region, and various abiotic
parameters of the sediment were assessed, such as the concentration of
organic matter, heavy metals and the distribution of particle size
fractions. Similarity analysis of invertebrate assemblages larger than
0.5 mm detected signs of disturbance in some areas, with an increased
number of pollution-tolerant and opportunistic species. Regression
models highlighted which habitat variables have the most impact on the
structure of communities.

The second chapter looks at the ecological status of ecosystems based on
the composition of benthic communities. Sixteen indicators of ecological
status were selected through a literature review, divided into three
categories according to their methodology: measures of abundance,
community diversity and indicator species. These indicators were applied
using the species lists obtained in the previous chapter, and the
majority detected diverse communities with no obvious sign of
disturbance. In addition, several significant correlations were detected
between indicators and habitat parameters, especially with heavy metal
concentrations. Each category of indicator provides important
information on the state of the ecosystem while presenting limitations,
in particular about reference conditions used to define ecological
status.

The last chapter considered human activities influencing the ecosystem,
in order to calculate a local anthropogenic footprint on communities
according to exposure gradients. A particle diffusion model for each
activity considered (aquaculture, dredging, industrial influence, city
influence, fisheries, sewage discharges, shipping) was developed using
the distance from their sources and physical factors. Several links were
discovered between the exposure indices obtained and the distribution of
benthic invertebrates, using predictive models
\textit{Hierarchical Modeling of Species Communities}. The cumulative
exposure index revealed areas of superposition of human activity. The
profile of the communities present in these zones is not particularly
disturbed, which corroborates the results of the previous chapters on
the status of the ecosystems considered.

This PhD thesis contributes to improving ecological knowledge in
sub-Arctic coastal ecosystems, in particular by presenting the first
benthic biodiversity census in the Sept-Îles region. Methods for
assessing ecological status and anthropogenic exposure have been
developed at the local scale (\textless100 km), which constitute
particularly interesting tools for managers in order to define
management targets and support conservation initiatives.

\begin{quote}
Keywords: marine ecology, sub-Arctic coastal ecosystems, benthic
invertebrates, biodiversity, community prediction, human activities,
anthropogenic exposure, ecological status assessment.
\end{quote}


  % [L'abstract doit être une traduction anglaise fidèle et grammaticalement correcte du résumé en français.]

\end{singlespace}
\cleardoublepage




% ----------------------------------------------------------------------%
% 5- Table des matières.                                               %
% ----------------------------------------------------------------------%

\tabledesmatieres



% ----------------------------------------------------------------------%
% 6- Liste des tableaux.                                               %
% ----------------------------------------------------------------------%

\listedestableaux

% ----------------------------------------------------------------------%
% 7- Table des matières.                                               %
% ----------------------------------------------------------------------%

\listedesfigures

% ----------------------------------------------------------------------%
% 8- Liste des abréviations (optionnel).                               %
% ----------------------------------------------------------------------%

% \listeabrev
% \begin{liste}
%
% \item[DOI]~: \textit{Digital Object Identifier}; identifiant numérique d'objet.
%
% \item[GIEC]~: Groupe d'experts Intergouvernemental sur l'Évolution du Climat.
%
% \item[IPBES]~: \textit{Intergovernmental Science-Policy Platform on Biodiversity and Ecosystem Services}; Plateforme intergouvernementale sur la biodiversité et les services écosystémiques.
%
% \end{liste}



% ----------------------------------------------------------------------%
% 9- Liste des symboles (optionnel).                                   %
% ----------------------------------------------------------------------%

% \listesymboles
% \begin{liste}
% \item[SYMBOLE 1] Ceci est la définition du symbole 1.
%
% \item[SYMBOLE 2] Ceci est la définition du symbole 2.
%
% \item[SYMBOLE 3] Ceci est la définition du symbole 3.
% \end{liste}

% ----------------------------------------------------------------------%
% Fin des liminaires.                                                  %
% ----------------------------------------------------------------------%

\cleardoublepage
