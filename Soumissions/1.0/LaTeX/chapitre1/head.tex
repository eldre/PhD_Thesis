\chapter{Évaluation de la biodiversité et de l'habitat des communautés benthiques côtiers en zone industrielle-portuaire subarctique}
\label{chap1}

\section{Résumé}
Les écosystèmes côtiers sont confrontés à des pressions anthropiques croissantes dans le monde entier et leur gestion nécessite une évaluation et une compréhension solides des impacts cumulatifs des activités humaines. Cette étude évalue la variation spatiale des communautés macrofauniques benthiques, des sédiments et des métaux lourds dans les écosystèmes côtiers subarctiques autour de Sept-Îles (Québec, Canada) - une zone portuaire importante dans le Golfe du Saint-Laurent. Les propriétés physiques des sédiments variaient dans la zone étudiée, avec un profil général sablo-vaseux, sauf à des endroits spécifiques de la Baie des Sept Îles où des concentrations plus élevées de matière organique et de métaux lourds ont été détectées. Les assemblages macrofauniques ont été évalués pour deux classes de taille de taxons (organismes > 0,5 mm et > 1 mm) et reliés aux paramètres de l'habitat à l'aide de modèles de régression. Des communautés d'organismes plus petits ont montré des signes de perturbation pour un assemblage proche des activités industrielles de la Baie des Sept Îles, avec un nombre accru d'espèces tolérantes et opportunistes, contrairement aux régions voisines dont la composition était similaire à celle d'autres écosystèmes dans le Golfe du Saint-Laurent. Cette étude améliore la compréhension des communautés benthiques subarctiques et contribue aux programmes de surveillance des écosystèmes en zone industrialo-portuaire. \linebreak[4]

L'article associé à ce chapitre, "\textit{Biodiversity and habitat assessment of coastal benthic communities in a sub-Arctic industrial harbour area}", a été rédigé en collaboration avec Christopher W. McKindsey, Cindy Grant, Lisa Tréau de Coeli, Richard St-Louis et Philippe Archambault. Il a été publié dans le journal \textit{Water}, dans la section spéciale \textit{Quantifying the Effects of Global Change on the Distribution and Quality of Aquatic Resources}, le 28 août 2020. J'ai établi les objectifs de ce chapitre avec Christopher W. McKindsey et Philippe Archambault, et j'ai effectué la collecte de données sur le terrain en 2016 et en 2017 avec le soutien de plusieurs stagiaires sous ma direction. J'ai compilé les bases de données et effectué les analyses statistiques, tout en intégrant les données et résultats de la campagne 2014 effectuée par Cindy Grant et Lisa Tréau de Coeli. J'ai dirigé la rédaction de l'article, où l'ensemble des co-auteurs a contribué à l'interprétation des résultats en fonction de leur expertise et à la révision générale. Les données liées à cet article sont accessibles dans le dépôt en ligne hébergé par le site Scholars Portal Dataverse avec l'identifiant unique [10.5683/SP2/5LJYXO](https://doi.org/10.5683/SP2/5LJYXO). Les résultats obtenus durant ces travaux ont été présentés lors de la Réunion Scientifique Annuelle de Québec-Océan à Rivière-du-Loup en novembre 2017, la \textit{World Conference on Marine biodiversity} à Montréal en mai 2018 et le Colloque International sur la Recherche Scientifique Industrielle-Portuaire à Sept-Îles en mai 2019.
\linebreak[4]

\begin{singlespace}
Dreujou E, McKindsey CW, Grant C, Tréau de Coeli L, St-Louis R, Archambault P (2020). Biodiversity and Habitat Assessment of Coastal Benthic Communities in a Sub-Arctic Industrial Harbor Area. \textit{Water} 12(9):2424. DOI:10.3390/w12092424.
\end{singlespace}

\textit{Les sections suivantes correspondent à celles de l'article publié.}
